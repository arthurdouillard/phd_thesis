\chapter{Introduction}
\label{chapter:introduction}

%\minitoc
\chapterwithfigures{\nameref*{chapter:introduction}}
%\chapterwithtables{\nameref*{chapter:introduction}}

\ifthenelse{\boolean{skipIntro}}{\endinput}{}

\section{Context}

\acf{AI} is a vast subject with several approaches whose common goal is to reproduce more or less
human behaviors. The field was arguably founded during the Dartmouth Workshop during the summer of
1956. From that time forward, rule-based systems, algorithms, and machine learning models were
developed to tackle a wide variety of domains, ranging from playing chess to understanding images
and text. Now in 2022, \ac{DL} research has underwent an almost exponential growth with important investments
from both research labs and private companies (\eg Google, Facebook, Heuritech, \etc). Applications
are widely available to the general public with improved web and search engines, automatic
translation, face recognition, speech recognition, autonomous driving, \etc.

A particular domain of \ac{AI} is \ac{CV} where a model must be able to automatically process images
and extract useful information. This domain is particularly important as the amount of images and
videos uploaded grow exponentially each year. This trove of data is hard to exploit due to the
unstructured nature of images. The best performing approaches of the 2000's were based on
handcrafted features such as SIFT \citep{lowe1999sift} or Fisher Vectors \todo{cit} followed by
machine learning models like Gradient Boosting \todo{cit}, \acs{SVM} \citep{cortes1995svm}, or
random forest \todo{cit}. From the year 2012, \acf{DL} models, a specific kind of machine learning
model, started to outperform previous approaches by a large margin. Specifically, a particular type
of \ac{DL} models called \acs{ConvNet}, well fitted for \ac{CV}, dominated the ILSVC competition
\citep{russakovsky2015imagenet_ilsvrc} with 2012's AlexNet architecture
\citep{krizhevsky2012alexnet}. All following state-of-the-art models without exception were also based on the
\ac{DL} paradigm.

\ac{DL} models are a succession of linear transformations and non-linear functions and are supposed
to be able to approximate any function \todo{universal approx}. For example, the most simple \ac{DL}
model, a \ac{MLP} with a single hidden layer for classification can be defined likewise:
%
\begin{equation}
      \hat{\vy} = f_\theta(\vx) = \operatorname{softmax}(\vW_o + \sigma(\vW_h \vx + \vb_h) + \vb_o))\,,
      \label{eq:intro_mlp}
\end{equation}
%
with $\vW_h \in \mathbb{R}^{H \times D}$, $\vb_h \in \mathbb{R}^{H}$,
$\vW_o \in \mathbb{R}^{C \times H}$, $\vb_o \in \mathbb{R}^{C}$ being the parameters and of the
network. $\vx \in \mathbb{R}^D$ the input data as a vector, and $\tilde{\vy} \in \mathbb{R}^C$ the
predicted probabilities per classes. $\sigma$ is a hidden non-linear activation, often a \ac{ReLU}
($\operatorname{ReLU(x)} = \text{max}(0, x)$), and $\operatorname{softmax}(\tilde{\vy}) =
      \nicefrac{e^{\tilde{\vy}}}{\sum_{i} \tilde{\vy}_i}$ the final non-linear activation. This small
neural network is trained to minimize a loss function. In classification, the most common is the
cross-entropy:
%
\begin{equation}
      \mcL(\hat{\vy}, \vy) = -\sum_i y_i \log \hat{y}_i\,,
      \label{eq:intro_ce}
\end{equation}
%
with $\vy$ a one-hot vector of the labels. Finally, to optimize the parameters of the neural
network, we often use the mini-batch \ac{SGD} algorithm or a variation thereof:

\begin{algorithm}
      \begin{algorithmic}[1]
            \Statex \textbf{input:} a dataset $\mathbb{D}$ with pairs of $(\vx, \vy)$
            \Statex \textbf{input:} a loss function $\mcL(\hat{\vy}, \vy)$
            \Statex \textbf{input:} a model function $f_\theta$
            \Statex \textbf{input:} a learning rate $\eta$ and a batch size $b$
            \Statex

            \While{stopping criterion not satisfied}
            \State $\vx$, $\vy$ $\gets$ sample mini-batch from $\mathbb{D}$
            \State Forward pass: $\hat{\vy}$ $\gets$ $f_\theta(\vx)$
            \State Compute loss: $\mcL$ $\gets$ $\mcL(\hat{\vy}, \vy)$
            \State Compute the gradients: $\delta$ $\gets$ $\nabla_\theta \mcL$
            \State Update all parameters: $\theta$ $\gets$ $\theta - \eta \delta$
            \EndWhile
      \end{algorithmic}
      \caption{Task procedure of the Ghost model}
      \label{algo:intro_sgd}
\end{algorithm}

Almost all \ac{DL} models, in \ac{CV}, \ac{NLP}, or even Speech use a form of gradient descent to
optimize differentiable functions.

In 2022, a majority of \ac{DL} models for computer vision are based on learned convolutions.
Handcrafted convolutions can be used to extract crude patterns such as edges \citep{lowe1999sift}.
Early works proposed to learn the convolution kernels
\citep{fukushima1980neocognitron,lecun1999lenet} as parameters of the neural networks. Those networks
are then called \ac{CNN}. In 2012, thanks to a large dataset and more efficient code working on
\acs{GPU}, \cite{krizhevsky2012alexnet} won the ILSVC competition
\citep{russakovsky2015imagenet_ilsvrc} where they had to classify a large dataset ---ImageNet---
made of 1M2 training images among 1000 classes. From that point forward, multiple improvements were
made to \ac{CNN} \citep{ioffe2015batchnorm,he2016resnet} and these methods have been applied not only
to classification but also object detection \citep{ren20fasterrcnn}, semantic segmentation
\citep{chen2018deeplab}, visual question answering \citep{benyounes2017mutan}, \etc.

\section{Motivations for Continual Learning}

Usually, when training a \ac{CNN}, we assume the dataset is immutable and \textit{i.i.d.}: no new
image nor new classes will be learned. The knowledge acquired on one dataset A can be
\textit{transferred} to another dataset B with different classes using \textbf{transfer learning}
\citep{razavian2014transferlearning}. However, in that case, the new model, while being efficient on
the dataset B, cannot classify the classes of dataset A.

\textbf{Continual Learning} aims to learn a continually changing dataset without forgetting the
previous knowledge. \eg a model could learn at first to classify among 10 classes, then add 10 more
classes, \etc. Continually learning an increasing number of classes is easily doable if the data of
previous classes is accessible. However, it may not always be the case: privacy reasons are often
raised ---think medical data or federated learning's user data. In that situation, learning new
classes lead to a \textbf{catastrophic forgetting} \citep{robins1995catastrophicforgetting} of
previous classes.

\section{Outline}

\begin{itemize}
      \item \autoref{chapter:regularization}: \nameref{chapter:regularization}\\
            I first review the existing methods based on regularization for continual learning. While
            regularizing a model's probabilities is very efficient to reduce forgetting in large-scale
            datasets, there are few works considering constraints on intermediate features. I cover in this
            chapter two contributions aiming to regularize directly the latent space of \acs{ConvNet}. The
            first one, \acf{PODNet} aims to reduce the drift of spatial statistics between the old and new
            model, which in effect reduces drastically forgetting even on large amount of tasks. I show in a
            second part a complementary method where we avoid pre-emptively forgetting by allocating
            locations in the latent space for yet unseen future class.
      \item \autoref{chapter:segmentation}: \nameref{chapter:segmentation}\\
            Then, I describe a recent application of \acf{CIL} to semantic segmentation. I show that
            the very nature of \acf{CSS} offer new specific challenges, namely forgetting on large
            images and a background shift. We tackle the first problem by extending our distillation
            loss introduced in the previous chapter to multi-scales. The second problem is solved by
            an efficient pseudo-labeling strategy. Finally, we consider the common rehearsal learning,
            but applied this time to \ac{CSS}. I show that it cannot be used naively because of memory
            complexity and design a light-weight rehearsal that is even more efficient.
      \item \autoref{chapter:dynamic}: \nameref{chapter:dynamic}\\
            Finally, I consider a completely different approach to continual learning: dynamic networks
            where the parameters are extended during training to adapt to new tasks. Previous works on
            this domain are hard to train and often suffer from parameter count explosion. For the
            first time in continual computer vision, we propose to use the Transformer architecture:
            the model dimension mostly fixed and shared across tasks, with the exception of an
            expansion of learned task tokens. With an encoder/decoder strategy where the decoder
            forward is specialized by a task token, we show state-of-the-art robustness to forgetting
            while our memory and computational complexities barely grow.
\end{itemize}

\section{Related Publications}

This thesis is based on the materials published in the following papers:

\begin{itemize}
      \item \fullcite{douillard2020podnet}
      \item \fullcite{douillard2020ghost}
      \item \fullcite{douillard2020plop}
      \item \fullcite{douillard2021objectrehearsal}
      \item \fullcite{douillard2021dytox}
      \item \fullcite{douillardlesort2021continuum}
\end{itemize}
