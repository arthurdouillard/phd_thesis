\chapter{Dynamic Strategy with Transformers}
\label{chapter:dynamic}

\begin{chapabstract}
    Deep network architectures struggle to continually learn new tasks without forgetting the
    previous tasks. A recent trend indicates that dynamic architectures based on an expansion of the
    parameters can reduce catastrophic forgetting efficiently in continual learning. However,
    existing approaches often require a task identifier at test-time, need complex tuning to balance
    the growing number of parameters, and barely share any information across tasks. As a result,
    they struggle to scale to a large number of tasks without significant overhead.\\
    In this paper, we propose a transformer architecture based on a dedicated encoder/decoder
    framework. Critically, the encoder and decoder are shared among all tasks. Through a dynamic
    expansion of special tokens, we specialize each forward of our decoder network on a task
    distribution. Our strategy scales to a large number of tasks while having negligible memory and
    time overheads due to strict control of the expansion of the parameters. Moreover, this
    efficient strategy doesn't need any hyperparameter tuning to control the network's expansion.
    Our model reaches excellent results on CIFAR100 and state-of-the-art performances on the
    large-scale ImageNet100 and ImageNet1000 while having fewer parameters than concurrent dynamic
    frameworks.

    \begin{itemize}
        \item \fullcite{douillard2021dytox}
    \end{itemize}

\end{chapabstract}
\newpage

\minitoc
\chapterwithfigures{\nameref*{chapter:dynamic}} \chapterwithtables{\nameref*{chapter:dynamic}}

\ifthenelse{\boolean{skipDyna}}{\endinput}{}


\section{Introduction}
\label{sec:dytox_intro}


Recent works
\citep{yoon2018dynamically_expandable_networks,li2019learning_to_grow,hung2019cpg,fernando2017path_net,golkar2019neural_pruning,serra2018hat}
dynamically expand the network architectures
\citep{yoon2018dynamically_expandable_networks,li2019learning_to_grow} or re-arrange their internal
structures \citep{fernando2017path_net,serra2018hat,hung2019cpg,golkar2019neural_pruning}.
Unfortunately at test-time, they require to know the task to which the test sample belongs --- in
order to know which parameters should be used. More recently, DER \citep{yan2021der} and Simple-DER
\citep{li2021preserve} discarded the need for this task identifier by learning a single classifier on
the concatenation of all produced embeddings by different subsets of parameters. Yet, these
strategies induce dramatic memory overhead when tackling a large number of tasks, and thus need
complex pruning as post-processing.

To improve the ease of use of continual learning frameworks for real-world applications, we aim to
design a dynamically expandable representation (almost) `for free' by having the three following
properties: \#1 \textbf{limited memory overhead} as the number of tasks grows, \#2 \textbf{limited
    time overhead} at test time and \#3 \textbf{no setting-specific hyperparameters} for improved
robustness when faced to an unknown (potentially large) number of tasks.

To this end, we leverage the computer vision transformer ViT \citep{dosovitskiy2020vit}. Transformers
\citep{vaswani2017transformer} offer a very interesting framework to satisfy the previously
mentioned constraints. Indeed, we build upon this architecture to design a \textbf{encoder/decoder
    strategy}: the encoder layers are shared among all members of our dynamic network; the unique
decoder layer is also shared but its forward pass is specialized by a \textbf{task-specific learned
    token} to produce task-specific embeddings. Thus, the memory growth of the dynamic network is
extremely limited: only a 384d vector per task, validating property \#1. Moreover, this requires no
hyperparameter tuning (property \#3). Finally, the decoder is explicitly designed to be
computationally lightweight (satisfying property \#2). We nicknamed our framework, DyTox, for
\textbf{DYnamic TOken eXpansion}. To the best of our knowledge, we are the first to apply the
transformer architecture to continual computer vision.

Our strategy is robust to different settings, and can easily scale to a large number of tasks. In
particular, we validate the efficiency of our approach on CIFAR100, ImageNet100, and ImageNet1000
(displayed on \autoref{fig:dytox_imagenet1000}) for multiple settings. We reach state-of-the-art results,
with only a small overhead thanks to our efficient dynamic strategy.

\begin{figure*}
    \centering
    \includegraphics[width=0.95\textwidth]{images/dytox/dytox.pdf}
    \caption{\textbf{DyTox transformer model}. An image is first split into multiple patches,
    embedded with a linear projection. The resulting patch tokens are processed by 5 successive
    Self-Attention Blocks (SAB) (\autoref{sec:related_cv}). For each task ($t = 1\dots T$), the processed
    patch tokens are then given to the Task-Attention Block (TAB) (\autoref{sec:tab}): each forward
    through the TAB is modified by a different task-specialized token $\theta_t\, \text{for}\, t \in
        \{1 \dots T\}$ (\autoref{sec:ensembling_cab}). The $T$ final embeddings are finally given
    separately to independent classifiers $\text{Clf}_t$ each predicting their task's classes $C^t$.
    All $|C^{1:T}|$ logits are activated with a sigmoid. For example, at task $t=3$, one forward is
    done through the SABs and three task-specific forwards through the unique TAB.}
    \label{fig:dytox_model}
\end{figure*}


\section{Related work}
\label{sec:dytox_related}

\paragraph{Continual learning} models tackle the catastrophic forgetting of the old classes
\citep{thrun1998lifelonglearning,french1999catastrophicforgetting}. In computer vision, most of
continual learning strategies applied on large-scale datasets use rehearsal learning: a limited
amount of the training data of old classes is kept during training
\citep{robins1995catastrophicforgetting}. This data is usually kept in raw form (\textit{e.g.},
pixels) \citep{rebuffi2017icarl,castro2018end_to_end_inc_learn,chaudhry2019tinyepisodicmemories} but
can also be compressed \citep{hayes2020remind,iscen2020incrementalfeatureadaptation}, or trimmed
\citep{douillard2021objectrehearsal} to reduce memory overhead; others store only a model to generate
new samples of past classes
\citep{kemker2018fearnet,shin2017deep_generative_replay,lesort2019generative}. In addition, most
approaches aim at limiting the changes in the model when new classes are learned. These constraints
can be directly applied on the weights
\citep{kirkpatrick2017ewc,zenke2017synaptic_intelligence,aljundi2018MemoryAwareSynapses,chaudhry2018riemannien_walk},
intermediary features
\citep{hou2019ucir,dhar2019learning_without_memorizing_gradcam,peng2019m2kd,douillard2020podnet,douillard2020plop},
prediction probabilities
\citep{li2018lwf,rebuffi2017icarl,castro2018end_to_end_inc_learn,cermelli2020modelingthebackground},
or on the gradients \citep{lopezpaz2017gem,chaudhry2019AGEM,farajtabar2020ogd,saha2021gpm}. All these
constraint-based methods use the same static network architectures which doesn't evolve through
time, usually a ResNet \citep{he2016resnet}, a LeNet \citep{lecun1999lenet}, or a small MLP.

\paragraph{Continual dynamic networks} In contrast, our paper and others focus on designing
\textbf{dynamic architectures} that best handle a growing training distribution
\citep{yoon2018dynamically_expandable_networks,li2019learning_to_grow}, in particular by dynamically
creating (sub-)members each specialized in one specific task
\citep{fernando2017path_net,golkar2019neural_pruning,
    hung2019cpg,rusu2016progressive,routingnetworkcollier,wen2020batchensemble}. Unfortunately, previous
approaches often require the sample's task identifier at test-time to select the right subset of
parameters. We argue this is an unrealistic assumption in a real-life situation where new samples
could come from any task.
Recently, DER \citep{yan2021der} proposed a dynamic expansion of the representation by adding a new
feature extractor per task. All extractors' embeddings would then be concatenated and fed to a
unified classifier, discarding the need for a task identifier at test-time. To limit an explosion in
the number of parameters, they aggressively prune each model after each task using the HAT
\citep{serra2018hat} procedure. Unfortunately, the pruning is hyperparameter sensitive. Therefore,
hyperparameters are tuned differently on each experiment: for example, learning a dataset in 10
steps or in 50 steps use different hyperparameters. While being impracticable, it is also
unrealistic because the number of classes is not known in advance in a true continual situation.
Simple-DER \citep{li2021preserve} also uses multiple extractors, but its pruning method doesn't need
any hyperparameters; the negative counterpart is that Simple-DER controls less the parameter growth
(2.5x higher than a base model). In contrast, we propose a framework dedicated to continual learning
that seamlessly enables a task-dynamic strategy, efficient on all settings, without any
setting-dependant modification and at almost no memory overhead. We share early class-agnostic
\citep{olah2017feature} layers similarly to TreeNets \citep{lee2015treenet} and base our strategy on
the Transformer architecture.

\paragraph{Transformers} were first introduced for machine translation
\citep{vaswani2017transformer}, with the now famous self-attention. While the original transformer
was made of encoder and decoder layers, later transformers starting from BERT \citep{devlin2018bert}
used a succession of identical encoder blocks. Then, ViT \citep{dosovitskiy2020vit} proposed to apply
transformers to computer vision by using patches of pixels as tokens. Multiple recent works,
including DeiT \citep{touvron2021deit}, CaiT \citep{touvron2021cait}, ConVit \citep{dascoli2021convit},
and Swin \citep{liu2021swin}, improved ViT with architecture and training procedures modifications.
PerceiverIO \citep{jaegle2021perceiverio} proposed a general architecture whose output is adapted to
different modalities using specific learned tokens, and whose computation is reduced using a small
number of latent tokens. Despite being successful across various benchmarks, transformers have not
yet been considered for continual computer vision to the best of our knowledge. Yet, we don't use
the transformer architecture for its own sake, but rather because of the intrinsic properties of
transformers; in particular, the seminal encoder/decoder framework allows us to build an efficient
architecture with strong capabilities against catastrophic forgetting.

\section{DyTox transformer model}
\label{sec:dytox_model}


\label{sec:dytox_problem}


Our goal is to learn a unified model that will classify an increasingly growing number of classes,
introduced in a fixed amount of steps $T$. At a given step $t \in \{1 \dots T\}$, the model is
exposed to new data belonging to new classes. Specifically, it learns from samples $\{(x_i^t,
    y_i^t)\}_{i}$, where $x_i^t$ is the $i$-th image of this task $t$ and $y_i^t$ is the associated
label within the label set $\mcC^t$. All task label sets are exclusive: $\mcC^0 \cap \mcC^1 \dots
    \mcC^T = \emptyset$. The main challenge is that the data are fully available only temporarily:
following most previous works, only a few samples from previous tasks $\{1 \dots t-1\}$ are
available for training at step $t$ as rehearsing data. Yet, the model should remain able to classify
test data coming from all seen classes $\mcC^{1:t}$. A table of notations is provided in the
supplementary materials.

The \autoref{fig:dytox_model} displays our DyTox framework, which is made of several components
(SAB, TAB, and Task Tokens) that we describe in the following sections.

\subsection{Background}
\label{sec:dytox_vit}

The vision transformer \citep{dosovitskiy2020vit} has three main components: the patch tokenizer, the
encoder made of Self-Attention Blocks, and the classifier.

\begin{figure}
    \centering
    \includegraphics[width=0.45\textwidth]{images/dytox/sab.pdf}
    \caption{\textbf{The Self-Attention Block (SAB)} combines a Self-Attention (SA), two Layer
        Norms, and one MLP with a single hidden layer. As in a ResNet, two shortcuts are used with
        element-wise addition.\vspace{-2em}}
    \label{fig:dytox_sab}
\end{figure}

\paragraph{Patch tokenizer}The fixed-size input RGB image is cropped into $N$ patches of equal
dimensions and then projected with a linear layer to a dimension $D$. Both operations, the cropping
and projection, are done with a single 2D convolution whose kernel size is equal to its stride size.
The resulting tensor $x_0 \in \mathbb{R}^{N \times D}$ is extended with a learned class token
$x_{\text{cls}} \in \mathbb{R}^D$ resulting in a tensor of shape $\mathbb{R}^{(N+1) \times D}$.
Following \citep{gehring2017posembeddings}, a learned positional embedding $p \in \mathbb{R}^{(N+1)
        \times D}$ is added (element-wise).


\paragraph{Self-Attention (SA) based encoder}The tokens are fed to a stack of transformer blocks
that we denote here as Self-Attention Blocks (SABs):
\begin{equation}
    \begin{aligned}
        x_{l}^{\prime} & =x_{l}+\operatorname{SA}_l\left(\operatorname{Norm}_{l,1}\left(x_{l}\right)\right)\,,                  \\
        x_{l+1}        & =x_{l}^{\prime}+\operatorname{MLP}_l\left(\operatorname{Norm}_{l,2}\left(x_{l}^{\prime}\right)\right),
    \end{aligned}
    \label{eq:dytox_sa_block}
\end{equation}
with $\operatorname{SA}$ a Self-Attention layer \citep{vaswani2017transformer}, $\operatorname{Norm}$
a layer normalization \citep{ba2016layernorm}, and $\operatorname{MLP}$ a Multi-Layer Perceptron with
a single hidden layer. We repeat these operations for each SAB, from $l=1$ to $l=L$. The resulting
tensor (which keeps the same dimension after every block) is $x_{L} \in \mathbb{R}^{(N+1) \times
        D}$. We display a visual illustration of a SA Block in \autoref{fig:dytox_sab}.


\paragraph{Classifier} In the original vision transformer (ViT \citep{dosovitskiy2020vit}), a learned
vector called the ``\textit{class token}'' is appended to the patch tokens after the tokenizer. This
special class token, when processed after all the SABs, is given to a linear classifier with a
softmax activation to predict the final probabilities. However, more recent works, as CaiT
\citep{touvron2021cait}, propose instead to introduce the class token only at the ultimate or
penultimate SAB to improve classification performance.

\subsection{Task-Attention Block (TAB)}
\label{sec:dytox_tab}

Contrary to previous transformer architectures, we don't have a class token, but rather what we
nicknamed ``\textbf{task tokens}''; the learned token of the $i^{th}$ task is denoted $\theta_i$.
This special token will only be added at the last block. To exploit this task token, we define a new
attention layer, that we call the Task-Attention. It first concatenates the patch tokens $x_L$
produced by the ultimate SAB with a task token $\theta_i$:
%
\begin{equation}
    z_i = [\theta_i, x_L] \, \in \mathbb{R}^{(\mathbb{N} + 1) \times \mathbb{D}}\,.
    \label{eq:dytox_concat_cls_token}
\end{equation}
%
This is then given to the Task-Attention (TA), inspired by the Class-Attention of Touvron et al.
\citep{touvron2021cait}:
%
\begin{equation}
    \begin{aligned}
        Q_i & =W_{q} \theta_i\,,                                                      \\
        K_i & =W_{k} z_i\,,                                                           \\
        V_i & =W_{v} z_i\,,                                                           \\
        A_i & =\operatorname{Softmax}\left(Q_i \cdot K_i^{T} / \sqrt{d / h}\right)\,, \\
        O_i & = W_{o} A_i V_i+b_{o} \, \in \mathbb{R}^{1 \times \mathbb{D}}\,,
    \end{aligned}
    \label{eq:dytox_ca_layer}
\end{equation}
%
with $d$ being the embedding dimension, and $h$ the number of attention heads
\citep{vaswani2017transformer}. Contrary to the classical Self-Attention, the Task-Attention defines
its query ($Q_i$) only from the task-token $\theta_i$ without using the patch tokens $x_L$. The
Task-Attention Block (TAB) is then a variation of the SAB where the attention is a Task-Attention
(TA):
\begin{equation}
    \begin{aligned}
        c^{\prime}       & =c+\operatorname{TA}\left(\operatorname{Norm}_1\left(z\right)\right)\,,                    \\
        c^{\prime\prime} & =c^{\prime}+\operatorname{MLP}\left(\operatorname{Norm}_2\left(c^{\prime}\right)\right)\,.
    \end{aligned}
    \label{eq:dytox_ca_block}
\end{equation}
Overall, our new architecture can be summarized by the repetition of SA Blocks
$\{\operatorname{SAB}_l\}_{l=1}^{L}$ (defined in \autoref{eq:dytox_sa_block}) ended by a single TA Block
$\operatorname{TAB}$ (defined in \autoref{eq:dytox_ca_block}):
%
\begin{equation}
    e_i = \operatorname{TAB} \circ\, ([\theta_i,\, \operatorname{SAB}_{l=L} \circ\, ... \operatorname{SAB}_{l=1}(x_0)]) \in \mathbb{R}^D\,.
    \label{eq:dytox_cab_sab}
\end{equation}
%
The final embedding $e_i$ is fed to a classifier $\operatorname{clf}$ made of a
$\operatorname{Norm}_c$ and a linear projection parametrized by $\{W_c, b_c\}$:
%
\begin{equation}
    \tilde{y}_i = \operatorname{Clf}(e_i) = W_c \operatorname{Norm}_c(e_i) + b_c\,.
\end{equation}

\subsection{Dynamic task token expansion}
\label{sec:dytox_ensembling_tab}
We defined in the previous section our base network, made of a succession of SABs and ended by a
single TAB. As detailed, the TAB has two inputs: the patch tokens $x_L$ extracted from the image and
a learned task-token $\theta_i$. We'll now detail how our framework evolves in a continual situation
at each new step.

During the first step, there is only one task token $\theta_1$. At each new step, we propose to
expand our parameter space by creating a new task token while keeping the previous ones. Thus, after
$t$ steps, we have $t$ task tokens ($\theta_i\,\text{for}\, i\in \{1 \dots t\}$). Given an image $x$
--- belonging to any of the seen tasks $\{1\dots \, t\}$ --- our model tokenizes it into $x_0$, and
processes it through the multiple SABs: this outputs the patch tokens $x_L$. Finally, our framework
does as many forward passes through the TAB as there are tasks: critically, each TAB forward passes
is executed with a different task token $\theta_i$, resulting in different task-specific forwards,
each producing the task-specific embeddings $e_i$ (see \autoref{fig:dytox_model}):
%
\begin{equation}
    \begin{aligned}
         & e_1 = \operatorname{TAB}([\theta_1, x_L])\,, \\
         & e_2 = \operatorname{TAB}([\theta_2, x_L])\,, \\
         & \dots                                        \\
         & e_t = \operatorname{TAB}([\theta_t, x_L])\,. \\
    \end{aligned}
    \label{eq:dytox_multiple_tab}
\end{equation}
%
Rather than concatenating all embeddings $\{e_1, e_2, \dots, e_t\}$ together and feeding them to one
classifier, we leverage \textbf{task-specific classifiers}. Each classifier $\operatorname{clf}_i$
is made of a $\operatorname{Norm}_i$ and a linear projection parametrized by $\{W_i, b_i\}$, with
$W_i \in \mathbb{R}^{\mcC^i \times D}$ and $b \in \mathbb{R}^{\mcC^i}$. It takes as input its
task-specific embedding $e_i$ and returns:
%
\begin{equation}
    \hat{y}_i = \operatorname{Clf}_i(e_i) = \sigma(W_i \operatorname{Norm}_i e_i + b_i)\,,
\end{equation}
%
the predictions for the classes $y_i \in \mcC^i$, where $\sigma(x) = \nicefrac{1}{(1 + e^{-x})}$ is
the sigmoid activation. In comparison with the softmax activation, the element-wise sigmoid
activation reduces the overconfidence in recent classes. Consequently, the model is better
calibrated, which is an important attribute of continual model
\citep{belouadah2019il2m,wu2019bias_correction,zhao2020weightalignement}. The loss is the
binary-cross entropy. The independent classifiers paradigm coupled with the sigmoid activation and
binary cross-entropy loss exclude explicitly a late fusion \citep{ramachandram2017multimodalreview}
of the task embeddings resulting in more \textbf{specialized classifiers}.

\paragraph{The overall structure of the DyTox strategy} is illustrated in \autoref{fig:dytox_model}.
We also show in \autoref{algo:dytox_tab_ensemble_1-1_bce} the pseudo-code of a forward pass at test-time
after having learned the task $t$. Critically, the test image can belong to any of the previously
seen tasks $\{1 \,\dots \, t\}$. Our dynamic task token expansion is more efficient than a naive
parameter expansion that would create a new copy of the whole network for each new task. (1) Our
expansion is limited to a new task token per new task, which is only $d=384$ new parameters. This is
small compared to the total model size ($\approx$ 11 million parameters). The \textbf{memory
    overhead is thus almost null}. (2) The computationally intensive blocks (\textit{i.e.}, the SABs)
are executed only once despite learning multiple tasks. In contrast, the TAB has as many forwards as
there are tasks. Though, this induces minimal overhead because the \textbf{Task-Attention has a
    linear complexity w.r.t the number of patches} while the Self-Attention is quadratic. Therefore, the
time overhead is sub-linear. We quantitatively show this in \autoref{sec:dytox_exp}.

\begin{algorithm}[tb]
    \caption{DyTox's forward pass at step $t$}
    \label{algo:dytox_tab_ensemble_1-1_bce}
    \hspace*{\algorithmicindent} \textbf{Input:} $x_0$ (initial patch tokens), $y$ ( ground-truth
    labels) \\
    \hspace*{\algorithmicindent} \textbf{Output:} $\hat{y}_{1:t}$ (predictions for all classes of
    $\mcC^{1:t}$)
    \begin{algorithmic}[1]
        % \Require $x_0$ (initial patch tokens), $y$ ( ground-truth labels)
        \State $x_L \gets \operatorname{SAB}_{l=L} \circ ... \operatorname{SAB}_{l=1}(x_0)$
        \Comment{\autoref{sec:dytox_vit}}

        \For{\texttt{$i \gets 1$; $i \leq t$; $i{+}{+}$}} \State $e_i \gets \operatorname{TAB}([\theta_i,
                x_L])$ \Comment{\autoref{sec:dytox_tab}} \State $\hat{y}_i \gets \operatorname{Clf}_i(e_i)$
        \Comment{\autoref{sec:dytox_ensembling_cab}} \EndFor

        \State $\hat{y}_{1:t} \gets [\hat{y}_1,\, \dots,\, \hat{y}_{t}]$
    \end{algorithmic}
\end{algorithm}

\vspace{-1em}
\paragraph{Context} The current transformer paradigm starting from BERT \citep{devlin2018bert} and
continuing with ViT \citep{dosovitskiy2020vit} is based on a encoder+classifier structure.
Differently, our dynamic framework strays is a resurgence of the encoder/decoder structure of the
original transformer \citep{vaswani2017transformer}: the encoder is shared (both in memory and
execution) for all outputs. The decoder parameters are also shared, but its execution is
task-specific with each task token, with each forward akin to a task-specific expert chosen from a
mixture of experts \citep{masoudnia2014mixture}. Moreover, multi-tasks text-based transformers have
natural language tokens as an indicator of a task \citep{raffel2019t5} (\textit{e.g.} "summarize the
following text"), in our context of vision we used our defined task tokens as indicators.

\label{sec:dytox_training}

\vspace{-0.5em}
\paragraph{Losses} Our model is trained with three losses: (1) the classification loss
$\mathcal{L_\text{clf}}$, a binary-cross entropy, (2) a knowledge distillation
\citep{hinton2015knowledge_distillation} $\mathcal{L_\text{kd}}$ applied on the probabilities, and
(3) the divergence loss $\mathcal{L_\text{div}}$. The distillation loss helps to reduce forgetting.
It is arguably quite naive, and more complex distillation losses
\citep{selvaraju2017gradcam,hou2019ucir,douillard2020podnet} could further improve results. The
divergence loss, inspired from the ``auxiliary classifier'' of DER \citep{yan2021der}, uses the
current last task's embedding $e_t$ to predict ($|\mcC^t| + 1$) probabilities: the current last
task's classes $\mcC^t$ and an extra class representing all previous classes that can be encountered
via rehearsal. This classifier is discarded at test-time and encourages a better diversity among
task tokens. The total loss is:
%
\begin{equation}
    \mathcal{L} = (1 - \alpha) \mathcal{L_\text{clf}} + \alpha \mathcal{L_\text{kd}} + \lambda \mathcal{L_\text{div}}\,,
    \label{eq:dytox_final_loss}
\end{equation}
%
with $\lambda$ a hyperparameter set to $0.1$ for \textbf{all} experiments. $\alpha$ correspond to
the fraction of the number of old classes over the number of new classes
$\frac{|C^{1:t-1}|}{|C^{1:t}|}$ as done by \citep{zhao2020weightalignement}. Therefore, $\alpha$ is
automatically set; this removes the need to finely tune this hyperparameter.


\section{Experiments}
\label{sec:dytox_exp}

\subsection{Benchmarks \& implementation}

\paragraph{Benchmarks \& Metrics} We evaluate our model on CIFAR100 \citep{krizhevskycifar100},
ImageNet100 and ImageNet1000 \citep{deng2009imagenet} (descriptions in the supplementary materials)
under different settings.
%where for each the number of new classes seen per step is different.
The standard continual scenario in ImageNet has 10 steps: thus we add 10 new classes per step in
ImageNet100, and 100 new classes per step in ImageNet1000. In CIFAR100, we compare performances on
10 steps (10 new classes per step), 20 steps (5 new classes per step), and 50 steps (2 new classes
per step). In addition to the top-1 accuracy, we also compare the top-5 accuracy on ImageNet. We
report the ``\textit{Avg}'' accuracy which is the average of the accuracies after each step as
defined by \citep{rebuffi2017icarl}. We also report the final accuracy after the last step
(``\textit{Last}''). Finally, in our tables, ``\textit{\#P}'' denotes the parameters count in
million after the final step.

\begin{table}[t]
    \centering
    \begin{tabular}{@{}l|cc@{}}
        \hline
        Hyperparameter      & CIFAR                   & ImageNet \Tstrut\Bstrut \\
        \hline
        \# SAB              & \multicolumn{2}{c}{5}                             \\
        \# CAB              & \multicolumn{2}{c}{1}                             \\
        \# Attentions Heads & \multicolumn{2}{c}{12}                            \\
        Embed Dim           & \multicolumn{2}{c}{384}                           \\
        Input Size          & 32                      & 224                     \\
        Patch Size          & 4                       & 16                      \\
        \hline
    \end{tabular}
    \caption{\textbf{DyTox's architectures} for CIFAR and ImageNet. The only difference between the
        two architectures is the patch size, as the image sizes vary between datasets.}
    \label{tab:dytox_archi}
\end{table}


\begin{table*}[t]
    \centering
    %	\hspace{mm}
    \begin{adjustbox}{max width=\textwidth}
        \begin{tabular}{l|ccccc}
            \toprule[0.3mm]
                                                                                   & \multicolumn{5}{c}{ImageNet1000 10 steps}                                                                                                                               \\
            %\cmidrule{2-11}
                                                                                   & \multirow{2}{*}{\textbf{$\#$P}}           & \multicolumn{2}{c}{\textbf{top-1}} & \multicolumn{2}{c}{\textbf{top-5}}                                                     \\
            \cmidrule{3-6}
            \textbf{Methods}                                                       &                                           & \textbf{Avg}                       & \textbf{Last}                      & \textbf{Avg}            & \textbf{Last}           \\
            \hline
            ResNet18 joint                                                         & $11.68$                                   & -                                  & -                                  & -                       & $89.27$                 \\
            Transf. joint                                                          & 11.35                                     & -                                  & 73.58                              & -                       & 90.60                   \\
            \midrule
            LwF-MC \scriptsize{\citep{rebuffi2017icarl}}                           & 11.68                                     & -                                  & -                                  & 48.45                   & 25.06                   \\
            \textit{E2E} \scriptsize{\citep{castro2018end_to_end_inc_learn}}       & 11.68                                     & -                                  & -                                  & 72.09                   & 52.29                   \\
            \textit{Simple-DER} \scriptsize{\citep{li2021preserve}}                & 28.00                                     & 66.63                              & 59.24                              & 85.62                   & 80.76                   \\
            iCaRL \scriptsize{\citep{rebuffi2017icarl}}                            & $11.68$                                   & $38.40$                            & $22.70$                            & $63.70$                 & $44.00$                 \\
            BiC \scriptsize{\citep{wu2019bias_correction}}                         & $11.68$                                   & -                                  & -                                  & $84.00$                 & $73.20$                 \\
            WA \scriptsize{\citep{zhao2020weightalignement}}                       & $11.68$                                   & $65.67$                            & $55.60$                            & $86.60$                 & $81.10$                 \\
            DER w/o P \scriptsize{\citep{yan2021der}}                              & 116.89                                    & 68.84                              & 60.16                              & 88.17                   & 82.86                   \\ % no pruning
            \textcolor{gray}{$\text{DER}^\dagger$} \scriptsize{\citep{yan2021der}} & \textcolor{gray}{-}                       & \textcolor{gray}{66.73}            & \textcolor{gray}{58.62}            & \textcolor{gray}{87.08} & \textcolor{gray}{81.89} \\
            \hline
            DyTox                                                                  & 11.36                                     & \textbf{71.29}                     & \textbf{63.34}                     & \textbf{88.59}          & \textbf{84.49}          \\
            \hline
        \end{tabular}
    \end{adjustbox}
    \caption{\textbf{Results on the ImageNet-1000 dataset}, learned with 10 steps of
        respectively 10 and 100 new classes. E2E and Simple-DER
        results come from their respective papers, and used a different class
        ordering. Other results come from \citet{yan2021der}. The $\dagger$ symbol means that
        \citet{yan2021der} needed setting-sensitive hyperparameters. Moreover, its reported parameters
        count was an average over all steps (\citet{yan2021der} reported 14.52M on ImageNet1000): the
        final parameters count (necessarily higher) was not available.}
    \label{tab:dytox_imagenet}
\end{table*}

\begin{comment}
\begin{table*}[t]
    \centering
    %	\hspace{mm}
    \begin{tabular}{l|ccccc|ccccc}
        \toprule[0.3mm]
                                                                 & \multicolumn{5}{c|}{ImageNet100 10 steps} & \multicolumn{5}{c}{ImageNet1000 10 steps}                                                                                                                                                                                                                                                           \\
        \cmidrule{2-11}
                                                                 & \multirow{2}{*}{\textbf{$\#$P}}           & \multicolumn{2}{c}{\textbf{top-1}}        & \multicolumn{2}{c|}{\textbf{top-5}} & \multirow{2}{*}{\textbf{$\#$P}} & \multicolumn{2}{c}{\textbf{top-1}} & \multicolumn{2}{c}{\textbf{top-5}}                                                                                                         \\
        \cmidrule{3-6}
        \cmidrule{8-11}
        \textbf{Methods}                                         &                                           & \textbf{Avg}                              & \textbf{Last}                       & \textbf{Avg}                    & \textbf{Last}                      &                                    & \textbf{Avg}            & \textbf{Last}           & \textbf{Avg}            & \textbf{Last}           \\
        \hline
        ResNet18 joint                                           & $11.22$                                   & -                                         & -                                   & -                               & $95.10$                            & $11.68$                            & -                       & -                       & -                       & $89.27$                 \\
        Transf. joint                                            & 11.00                                     & -                                         & 79.12                               & -                               & 93.48                              & 11.35                              & -                       & 73.58                   & -                       & 90.60                   \\
        \midrule
        LwF-MC \cite{li2018lwf,rebuffi2017icarl}                 & 11.2                                      & -                                         & -                                   & 80.79                           & 66.43                              & 11.2                               & -                       & -                       & 48.45                   & 25.06                   \\
        \textit{E2E} \cite{castro2018end_to_end_inc_learn}       & 11.22                                     & -                                         & -                                   & 89.92                           & 80.29                              & 11.68                              & -                       & -                       & 72.09                   & 52.29                   \\
        \textit{Simple-DER} \cite{li2021preserve}                & -                                         & -                                         & -                                   & -                               & -                                  & 28.00                              & 66.63                   & 59.24                   & 85.62                   & 80.76                   \\
        iCaRL \cite{rebuffi2017icarl}                            & $11.22$                                   & -                                         & -                                   & $83.60$                         & $63.80$                            & $11.68$                            & $38.40$                 & $22.70$                 & $63.70$                 & $44.00$                 \\
        BiC \cite{hou2019ucir}                                   & $11.22$                                   & -                                         & -                                   & $90.60$                         & $84.40$                            & $11.68$                            & -                       & -                       & $84.00$                 & $73.20$                 \\
        WA \cite{zhao2020weightalignement}                       & $11.22$                                   & -                                         & -                                   & $91.00$                         & $84.10$                            & $11.68$                            & $65.67$                 & $55.60$                 & $86.60$                 & $81.10$                 \\
        RPSNet \cite{rajasegaran2019rpsnet}                      &                                           & -                                         & -                                   & $87.90$                         & $74.00$                            & -                                  & -                       & -                       & -                       & -                       \\
        DER w/o P \cite{yan2021der}                              & 112.27                                    & \textbf{77.18}                            & 66.70                               & \textbf{93.23}                  & 87.52                              & 116.89                             & 68.84                   & 60.16                   & 88.17                   & 82.86                   \\ % no pruning
        \textcolor{gray}{$\text{DER}^\dagger$} \cite{yan2021der} & \textcolor{gray}{-}                       & \textcolor{gray}{76.12}                   & \textcolor{gray}{66.06}             & \textcolor{gray}{92.79}         & \textcolor{gray}{88.38}            & \textcolor{gray}{-}                & \textcolor{gray}{66.73} & \textcolor{gray}{58.62} & \textcolor{gray}{87.08} & \textcolor{gray}{81.89} \\
        \hline
        DyTox                                                    & 11.01                                     & \textbf{77.15}                            & \textbf{69.10}                      & 92.04                           & \textbf{87.98}                     & 11.36                              & \textbf{71.29}          & \textbf{63.34}          & \textbf{88.59}          & \textbf{84.49}          \\
        \hline
    \end{tabular}
    \caption{\textbf{Results on ImageNet-100  and ImageNet-1000 datasets}, learned with 10 steps of
        respectively 10 and 100 new classes. E2E \cite{castro2018end_to_end_inc_learn} and Simple-DER
        \cite{li2021preserve} results come from their respective papers, and used a different class
        ordering. Other results come from \cite{yan2021der}. The $\dagger$ symbol means that
        \cite{yan2021der} needed setting-sensitive hyperparameters. Moreover, its reported parameters
        count was an average over all steps (\cite{yan2021der} reported 14.52M on ImageNet1000): the
        final parameters count (necessarily higher) was not available.}
    \label{tab:dytox_imagenet}
\end{table*}
\end{comment}



\paragraph{Implementation details} As highlighted in \autoref{tab:archi}, our network has the same
structure across all tasks. Specifically, we use 5 Self-Attention Blocks (SABs), 1 Task-Attention
Block (TAB). All 6 have an embedding dimension of 384 and 12 attention heads. We designed this
shallow transformer to have a comparable parameters count to other baselines, but also made it wider
than usual "tiny" models \citep{dosovitskiy2020vit,touvron2021deit,touvron2021cait}. We tuned all
hyperparameters for CIFAR100 with 10 steps on a validation set made of 10\% of the training set, and
then kept them fixed for all other settings, ImageNet included. The only difference between the two
datasets is that ImageNet images are larger; thus the patch size is larger, and overall the base
transformer has slightly more parameters on ImageNet than on CIFAR (11.00M vs 10.72M) because of a
bigger positional embedding. We use the attention with spatial prior (introduced by ConViT
\citep{dascoli2021convit}) for all SABs which allows training transformers on a small dataset (like
CIFAR) without pretraining on large datasets or complex regularizations. Following previous works
\citep{rebuffi2017icarl,yan2021der}, we use for all models (baselines included) 2,000 images of
rehearsal memory for CIFAR100 and ImageNet100, and 20,000 images for ImageNet1000. The
implementations of the continual scenarios are provided by Continuum
\citep{douillardlesort2021continuum}. Our network implementation is based on the DeiT
\citep{touvron2021deit} code base which itself uses extensively the timm library
\citep{wightman2019timm}. The code is released
publicly\footnote{\footnotesize{\url{https://github.com/arthurdouillard/dytox}.}}. The full
implementation details are in the appendix.


\begin{figure}
    \centering
    \begin{subfigure}{.5\textwidth}
        \centering
        \includegraphics[width=0.9\textwidth]{images/dytox/imagenet1000.png}
        \caption{\textbf{ImageNet1000}}
        \label{fig:dytox_imagenet1000}
    \end{subfigure}%
    \begin{subfigure}{.5\textwidth}
        \centering
        \includegraphics[width=0.9\textwidth]{images/dytox/imagenet100.png}
        \caption{\textbf{ImageNet100}
        \label{fig:dytox_imagenet100}
    \end{subfigure}
    \caption{\textbf{Performance evolution on ImageNet-\{100, 1000\}.} The top-5 accuracy (\%) is
        reported after learning each task. Our model DyTox (in \textbf{\textcolor{red}{red}}) reaches
        state-of-the-parterformance while using significantly less parameters than concurrent models.
        Note that at the initial step before the continual process begins, our model has performance
        comparable to other baselines: the performance gain is achieved by reducing catastrophic
        forgetting.}
    \label{fig:dytox_imagenet}
\end{figure}


\begin{comment}

\begin{figure}
    \centering
    \includegraphics[width=0.45\textwidth]{images/dytox/imagenet1000.png}
    \caption{\textbf{DyTox's continual learning performance on ImageNet1000}: for each task, 100 new
        classes are learned while previously learned classes are not fully accessible but shouldn't be
        forgotten. Our strategy DyTox (in \textbf{\textcolor{red}{red}}) is state-of-the-art by a large
        margin. Note that at the initial step before the continual process begins (denoted by a dashed
        rectangle, our model has performance
        comparable to other baselines: the performance gain is achieved by reducing catastrophic
        forgetting. Moreover, we have systematically fewer parameters than previous approaches.}
    \label{fig:dytox_imagenet1000}
\end{figure}
% \includegraphics[width=1.9ex]{images/dytox/rectangle.png})

\begin{figure}
    \centering
    \includegraphics[width=0.40\textwidth]{images/dytox/imagenet100.png}
    \caption{\textbf{Performance evolution on ImageNet100}. The top-5 accuracy (\%) is reported
        after learning each task. Our model DyTox (in \textbf{\textcolor{red}{red}}) surpasses
        significantly most baselines, and is of equal performance as the complex DER that uses pruning
        with setting-specific hyperparameters.\vspace{-1em}}
    \label{fig:dytox_imagenet100}
\end{figure}
\end{comment}



\subsection{Quantitative results}

\paragraph{ImageNet}
We report performances in \autoref{tab:dytox_imagenet} on the complex ImageNet dataset. The $^\dagger$
marks the DER with setting-specific pruning, and DER w/o P is for the DER without pruning. In
ImageNet100, DyTox reaches 69.10\% and outperforms DER$^\dagger$ by +3.04 percentage points (\pp) in
``Last'' top-1 accuracy. Though, DyTox and DER w/o P somehow perform similarly in ``Avg'' accuracy
on this setup, as highlighted in the performance evolution displayed in \autoref{fig:dytox_imagenet100}.
Most importantly, on the larger-scale ImageNet1000, DyTox systematically performs best on all
metrics despite having lower parameters count. Specifically, DyTox reaches 71.29\% in ``Avg'' top-1
accuracy, and 63.34\% in ``Last'' top-1 accuracy. This outperforms the previous state-of-the-art DER
w/o P (68.84\% in ``Avg'', 60.16\% in ``Last'') which has 10 ResNet18 in parallel and 116.89M
parameters. Compared to the pruned DER$^\dagger$, DyTox has a +4.56 \pp in top-1 and a +1.51 \pp in
top-5 for the ``Avg'' accuracy. All models evolutions on ImageNet1000 are illustrated in
\autoref{fig:dytox_imagenet1000}: DyTox constantly surpasses previous state-of-the-art models --- despite
having a comparable performance at the first step and fewer parameters.

\begin{table*}[t]
    \centering
    \resizebox{1.0\textwidth}{!}{%
        \begin{tabular}{@{}l|ccc|ccc|ccc}
            \hline
                                                   & \multicolumn{3}{c}{10 steps} & \multicolumn{3}{c}{20 steps}                  & \multicolumn{3}{c}{50 steps}                                                                                                                                                                                                                   \\
            \textbf{Methods}                       & \textbf{\#P}                 & \textbf{Avg}                                  & \textbf{Last}                & \textbf{\#P}        & \textbf{Avg}                                  & \textbf{Last}                    & \textbf{\#P}        & \textbf{Avg}                                  & \textbf{Last}                    \\
            \hline
            ResNet18 Joint                         & 11.22                        & -                                             & 80.41                        & 11.22               & -                                             & 81.49                            & 11.22               & -                                             & 81.74                            \\
            Transf. Joint                          & 10.72                        & -                                             & 76.12                        & 10.72               & -                                             & 76.12                            & 10.72               & -                                             & 76.12                            \\
            \hline
            iCaRL                                  & 11.22                        & 65.27\scriptsize{\mypm1.02}                   & 50.74                        & 11.22               & 61.20\scriptsize{\mypm0.83}                   & 43.75                            & 11.22               & 56.08\scriptsize{\mypm0.83}                   & 36.62                            \\
            UCIR                                   & 11.22                        & 58.66\scriptsize{\mypm0.71}                   & 43.39                        & 11.22               & 58.17\scriptsize{\mypm0.30}                   & 40.63                            & 11.22               & 56.86\scriptsize{\mypm0.83}                   & 37.09                            \\
            BiC                                    & 11.22                        & 68.80\scriptsize{\mypm1.20}                   & 53.54                        & 11.22               & 66.48\scriptsize{\mypm0.32}                   & 47.02                            & 11.22               & 62.09\scriptsize{\mypm0.85}                   & 41.04                            \\
            WA                                     & 11.22                        & 69.46\scriptsize{\mypm0.29}                   & 53.78                        & 11.22               & 67.33\scriptsize{\mypm0.15}                   & 47.31                            & 11.22               & 64.32\scriptsize{\mypm0.28}                   & 42.14                            \\
            PODNet                                 & 11.22                        & 58.03\scriptsize{\mypm1.27}                   & 41.05                        & 11.22               & 53.97\scriptsize{\mypm0.85}                   & 35.02                            & 11.22               & 51.19\scriptsize{\mypm1.02}                   & 32.99                            \\
            RPSNet                                 & 56.5\,\,                     & 68.60                                         & 57.05                        & -                   & -                                             & -                                & -                   & -                                             & -                                \\
            DER \small{w/o P}                      & 112.27                       & 75.36\scriptsize{\mypm0.36}                   & \textbf{65.22}               & 224.55              & 74.09\scriptsize{\mypm0.33}                   & 62.48                            & 561.39              & 72.41\scriptsize{\mypm0.36}                   & 59.08                            \\ % no pruning
            \textcolor{gray}{$\text{DER}^\dagger$} & \textcolor{gray}{-}          & \textcolor{gray}{74.64\scriptsize{\mypm0.28}} & \textcolor{gray}{64.35}      & \textcolor{gray}{-} & \textcolor{gray}{73.98\scriptsize{\mypm0.36}} & \textcolor{gray}{\textbf{62.55}} & \textcolor{gray}{-} & \textcolor{gray}{72.05\scriptsize{\mypm0.55}} & \textcolor{gray}{\textbf{59.76}} \\
            \hline
            DyTox                                  & 10.73                        & 73.66\mysmpm{0.02}                            & 60.67\mysmpm{0.34}           & 10.74               & 72.27\mysmpm{0.18}                            & 56.32\mysmpm{{0.61}}             & 10.77               & 70.20\mysmpm{0.16}                            & 52.34\mysmpm{0.26}               \\
            DyTox+                                 & 10.73                        & \textbf{75.54}\mysmpm{0.10}                   & 62.06\mysmpm{0.25}           & 10.74               & \textbf{75.04}\mysmpm{0.11}                   & 60.03\mysmpm{0.45}               & 10.77               & \textbf{74.35}\mysmpm{0.05}                   & 57.09\mysmpm{0.13}               \\
            \hline
        \end{tabular}
    }
    \caption{\textbf{Results on CIFAR100} averaged over three different class orders. Baselines
        results are come from \cite{yan2021der}. The $\dagger$ symbol means that \cite{yan2021der}
        needed setting-sensitive hyperparameters. Moreover, its reported parameters count was an average
        over all steps: the final parameters count (necessarily higher) was not available.}
    \label{tab:dytox_cifar100-b0}
\end{table*}


\begin{figure*}[t!]
    \centering
    \includegraphics[width=1.0\linewidth]{images/dytox/cifar.png}
    \caption{\textbf{Performance evolution on CIFAR100}. The top-1 accuracy (\%) is reported after
        learning each task. \textbf{Left} is evaluated with 10 steps, \textbf{middle} with 20 steps, and
        \textbf{right} with 50 steps.}
    \label{fig:dytox_increment_cifar}
\end{figure*}

DyTox is able to scale correctly while handling seamlessly the parameter growth by sharing most of
the weights across tasks. In contrast, DER had to propose a complex pruning method; unfortunately,
this pruning required different hyperparameter values for different settings. Despite this, the
pruning in DER$^\dagger$ is less efficient when classes diversity increase: DER$^\dagger$ doubles in
size between ImageNet100 and ImageNet1000 (\citep{yan2021der} reports 7.67M \textit{vs.} 14.52M)
while handling the same amount of tasks (10). Note that these parameter counts reported for
DER$^\dagger$ in \citep{yan2021der} are in fact averages over all steps: the final parameters count
(necessarily higher) was not available and thus is not reported in our tables. Simple-DER also
applies pruning but without hyperparameter tuning; while simpler, the pruning is also less efficient
and induces larger model (28.00M parameters).

\vspace{-0.5em}
\paragraph{CIFAR100} \autoref{tab:dytox_cifar100-b0} shows results for all approaches on CIFAR100. The
more steps there are, the larger the forgetting is and thus the lower the performances are. Those
settings are also displayed in \autoref{fig:dytox_increment_cifar} after each task. In every setting,
DyTox is close to DER w/o P  for much fewer parameters (up to 52x less). Critically, DyTox is
significantly above other baselines: \textit{e.g.} DyTox is up to +25\% in ``Last'' accuracy in the
50 steps setup.

\vspace{-1em}
\paragraph{Improved training procedure} To bridge the gap between DyTox and DER w/o P on CIFAR100,
we introduce a new efficient training procedure for continual learning. Using MixUp
\citep{hingyi2018mixup}, we linear interpolate new samples with existing samples. The interpolation
factor $\lambda \sim \operatorname{Beta}(\alpha, \alpha)$ is sampled with $\alpha=0.8$: the pixels
of two images are mixed ($x = \lambda x_1 + (1 - \lambda) x_2$) as their labels ($y = \lambda y_1 +
    (1 - \lambda) y_2$). MixUp was shown to have two main effects: (1) it diversifies the training
images and thus enlarges the training distribution on the vicinity of each training sample
\citep{chapelle2001vicinalrisk} and (2) it improves the network calibration
\citep{guo2017miscalibration,thulasidasan2019mixupcalibration}, reducing the overconfidence in recent
classes. Thus MixUp has shared motivation with the sigmoid activation. When DyTox is combined with
this MixUp procedure, nicknamed as DyTox+, this significantly improves the state-of-the-art in
``Avg'' accuracy in all three settings of \autoref{tab:dytox_cifar100-b0}. We also provide in the appendix
further improvement for this new continual training procedure providing even larger gain on both
CIFAR100 and ImageNet100.

\subsection{Model introspection on CIFAR100}

\paragraph{Memory overhead}
We only add a vector of size $d=384$ per task; thus, the overhead in memory (not considering the
growing classifier which is common for all continual models) is only of $+0.004\%$ per step. Even in
the challenging setting of CIFAR100 with 50 tasks, our memory overhead is almost null ($+0.2\%$).

\label{sec:dytox_comp_over}
\vspace{-1em}
\paragraph{Computational overhead} The vast majority of the computation is done in the SABs, thus
shared among all tasks. The dynamical component of our model is located at the ultimate TAB.
Moreover, the Task-Attention, contrary to the Self-Attention, has a time complexity linear in terms
of tokens and not quadratic reducing the time overhead to an acceptable sub-linear amount. Overall,
for each new task, one forward pass takes $2.24\%$ more time than for the base transformer.

\vspace{-1em}
\begin{table}[t]
    \centering
    \begin{tabular}{@{}l|c|cc}
        \hline
                          & 1 step                                          & \multicolumn{2}{c}{50 steps}                                                                             \\
        \textbf{Training} & \textbf{Last} ($\uparrow$)                      & \textbf{Last} ($\uparrow$)                      & \textbf{Forgetting} ($\downarrow$)\Tstrut\Bstrut       \\
        \hline
        DyTox             & 76.12                                           & 52.34                                           & 33.15 \Tstrut                                          \\
        DyTox+            & 77.51\scriptsize{\textcolor{OliveGreen}{+1.39}} & 57.09\scriptsize{\textcolor{OliveGreen}{+4.75}} & 31.50\scriptsize{\textcolor{OliveGreen}{-1.65}}\Bstrut \\
        \hline
    \end{tabular}
    \caption{\textbf{``Last'' accuracy and forgetting} \cite{chaudhry2018riemannien_walk} on
        CIFAR100 for the joint (1 step, no continual) and 50 steps settings.\vspace{-1em}}
    \label{tab:dytox_training_plus}
\end{table}


\paragraph{Training procedure introspection} Our DyTox+ strategy with MixUp really reduces
catastrophic forgetting and does not just improve raw performances. This is shown in
\autoref{tab:dytox_training_plus}, where we compare DyTox \textit{vs.} DyTox+ strategies on CIFAR100.
While MixUp only slightly improves by 1.39 \pp the accuracy in joint learning (no continual, 1
step), MixUp greatly improves the performance by 4.75 \pp in the 50 steps continual scenario. To
further illustrate this, we also report the Chaudhry et al.'s forgetting
\citep{chaudhry2018riemannien_walk} measure which compares how performances dropped compared to
previous steps. MixUp reduces this forgetting by 1.65 \pp.

\vspace{-1em}
\label{sec:dytox_ablations}
\paragraph{Model ablations} We ablate the importance of the different components of DyTox in
\autoref{tab:dytox_ablation}. We add on the base transformer a naive knowledge distillation
\citep{hinton2015knowledge_distillation} and a finetuning
\citep{castro2018end_to_end_inc_learn,hou2019ucir,douillard2020podnet,yan2021der} applied after each
task on a balanced set of new data and rehearsal data. Finally, our DyTox strategy exploits directly
the very nature of transformers (separated task information from the pixels information) to tackle
catastrophic forgetting with three components: (1) a task token expansion, (2) a divergence
classifier, and (3) independent classifiers. All three greatly improve over the baseline transformer
($42.21\% \rightarrow 52.34\%$ in ``Last'') while having almost no memory overhead ($+0.2\%$). The
divergence classifier improves the diversity between task tokens: we observed that the minimal
Euclidean distance between them increases by 8\%. Moreover, we also remarked that having independent
classifiers reduces the Chaudhry et al.'s forgetting \citep{chaudhry2018riemannien_walk} by more than
24\%.

\begin{table}
    \centering
    \begin{tabular}{ll|ccccc|cc}
         &                                                                                               & \rot{\footnotesize{Knowledge Distillation}} & \rot{\footnotesize{Finetuning}} & \rot{\footnotesize{Token Expansion}} & \rot{\footnotesize{Divergence Classifier}} & \rot{\footnotesize{Indendepent Classifiers}} & \textbf{Avg} & \textbf{Last} \\
        \hline
        \parbox[t]{2mm}{\multirow{7}{*}{\rotatebox[origin=c]{90}{\textbf{DyTox}}}}
         & \parbox[t]{3mm}{\multirow{3}{*}{\rotatebox[origin=c]{90}{\textbf{\scriptsize{Transformer}}}}}
         &                                                                                               &                                             &                                 &                                      &                                            & 60.69                                        & 38.87\Tstrut                 \\
         &                                                                                               & \cmark                                      &                                 &                                      &                                            &                                              & 61.62        & 39.35         \\
         &                                                                                               & \cmark                                      & \cmark                          &                                      &                                            &                                              & 63.42        & 42.21         \\[3pt]
        %\vspace{0.01cm}
        \cline{2-9}
         & \parbox[t]{3mm}{\multirow{3}{*}{\rotatebox[origin=c]{90}{\textbf{\scriptsize{Dynamic}}}}}
         & \cmark                                                                                        & \cmark                                      & \cmark                          &                                      &                                            & 67.30                                        & 47.57\Tstrut                 \\
         &                                                                                               & \cmark                                      & \cmark                          & \cmark                               & \cmark                                     &                                              & 68.28        & 49.45         \\
         &                                                                                               & \cmark                                      & \cmark                          & \cmark                               & \cmark                                     & \cmark                                       & 70.20        & 52.34\Bstrut  \\
        \hline
    \end{tabular}
    \caption{\textbf{Ablations} of the different key components of our DyTox architecture. We report
    the average accuracy and the last accuracy on CIFAR100 for the setting with 50
    steps.\vspace{-1em}}
    \label{tab:dytox_ablation}
\end{table}



\section{Conclusion}

In this paper, we propose DyTox, a new dynamic strategy for continual learning based on transformer
architecture. In our model, self-attention layers are shared across all tasks, and we add
task-specific tokens to achieve task-specialized embeddings through a new task-attention layer.
This architecture allows us to dynamically process new tasks with very little memory overhead and
does not require complex hyperparameter tuning. Our experiments show that our framework scales to
large datasets like ImageNet1k with state-of-the-art performances. Moreover, when a large number of
tasks is considered (\textit{i.e.} CIFAR100 50 steps) our number of parameters increases reasonably
contrary to previous dynamic strategies.

\vspace{0.1cm}\noindent \textbf{Limitations:} True continual learning aims at learning an almost
unlimited number of tasks with low forgetting. No current approaches are yet able to do so. Thus,
forgetting is not yet solved for continual learning but our model is a step forward in that
direction.

\noindent \textbf{Broader impact:} Machine learning models often are biased, with some classes
suffering from lower performances. Studying forgetting in continual learning provides insights about
the difference in performances between classes. Our task-specialized model could help reduce these
biases.
